% Options for packages loaded elsewhere
\PassOptionsToPackage{unicode}{hyperref}
\PassOptionsToPackage{hyphens}{url}
%
\documentclass[
]{article}
\usepackage{amsmath,amssymb}
\usepackage{iftex}
\ifPDFTeX
  \usepackage[T1]{fontenc}
  \usepackage[utf8]{inputenc}
  \usepackage{textcomp} % provide euro and other symbols
\else % if luatex or xetex
  \usepackage{unicode-math} % this also loads fontspec
  \defaultfontfeatures{Scale=MatchLowercase}
  \defaultfontfeatures[\rmfamily]{Ligatures=TeX,Scale=1}
\fi
\usepackage{lmodern}
\ifPDFTeX\else
  % xetex/luatex font selection
\fi
% Use upquote if available, for straight quotes in verbatim environments
\IfFileExists{upquote.sty}{\usepackage{upquote}}{}
\IfFileExists{microtype.sty}{% use microtype if available
  \usepackage[]{microtype}
  \UseMicrotypeSet[protrusion]{basicmath} % disable protrusion for tt fonts
}{}
\makeatletter
\@ifundefined{KOMAClassName}{% if non-KOMA class
  \IfFileExists{parskip.sty}{%
    \usepackage{parskip}
  }{% else
    \setlength{\parindent}{0pt}
    \setlength{\parskip}{6pt plus 2pt minus 1pt}}
}{% if KOMA class
  \KOMAoptions{parskip=half}}
\makeatother
\usepackage{xcolor}
\usepackage[margin=1in]{geometry}
\usepackage{color}
\usepackage{fancyvrb}
\newcommand{\VerbBar}{|}
\newcommand{\VERB}{\Verb[commandchars=\\\{\}]}
\DefineVerbatimEnvironment{Highlighting}{Verbatim}{commandchars=\\\{\}}
% Add ',fontsize=\small' for more characters per line
\usepackage{framed}
\definecolor{shadecolor}{RGB}{248,248,248}
\newenvironment{Shaded}{\begin{snugshade}}{\end{snugshade}}
\newcommand{\AlertTok}[1]{\textcolor[rgb]{0.94,0.16,0.16}{#1}}
\newcommand{\AnnotationTok}[1]{\textcolor[rgb]{0.56,0.35,0.01}{\textbf{\textit{#1}}}}
\newcommand{\AttributeTok}[1]{\textcolor[rgb]{0.13,0.29,0.53}{#1}}
\newcommand{\BaseNTok}[1]{\textcolor[rgb]{0.00,0.00,0.81}{#1}}
\newcommand{\BuiltInTok}[1]{#1}
\newcommand{\CharTok}[1]{\textcolor[rgb]{0.31,0.60,0.02}{#1}}
\newcommand{\CommentTok}[1]{\textcolor[rgb]{0.56,0.35,0.01}{\textit{#1}}}
\newcommand{\CommentVarTok}[1]{\textcolor[rgb]{0.56,0.35,0.01}{\textbf{\textit{#1}}}}
\newcommand{\ConstantTok}[1]{\textcolor[rgb]{0.56,0.35,0.01}{#1}}
\newcommand{\ControlFlowTok}[1]{\textcolor[rgb]{0.13,0.29,0.53}{\textbf{#1}}}
\newcommand{\DataTypeTok}[1]{\textcolor[rgb]{0.13,0.29,0.53}{#1}}
\newcommand{\DecValTok}[1]{\textcolor[rgb]{0.00,0.00,0.81}{#1}}
\newcommand{\DocumentationTok}[1]{\textcolor[rgb]{0.56,0.35,0.01}{\textbf{\textit{#1}}}}
\newcommand{\ErrorTok}[1]{\textcolor[rgb]{0.64,0.00,0.00}{\textbf{#1}}}
\newcommand{\ExtensionTok}[1]{#1}
\newcommand{\FloatTok}[1]{\textcolor[rgb]{0.00,0.00,0.81}{#1}}
\newcommand{\FunctionTok}[1]{\textcolor[rgb]{0.13,0.29,0.53}{\textbf{#1}}}
\newcommand{\ImportTok}[1]{#1}
\newcommand{\InformationTok}[1]{\textcolor[rgb]{0.56,0.35,0.01}{\textbf{\textit{#1}}}}
\newcommand{\KeywordTok}[1]{\textcolor[rgb]{0.13,0.29,0.53}{\textbf{#1}}}
\newcommand{\NormalTok}[1]{#1}
\newcommand{\OperatorTok}[1]{\textcolor[rgb]{0.81,0.36,0.00}{\textbf{#1}}}
\newcommand{\OtherTok}[1]{\textcolor[rgb]{0.56,0.35,0.01}{#1}}
\newcommand{\PreprocessorTok}[1]{\textcolor[rgb]{0.56,0.35,0.01}{\textit{#1}}}
\newcommand{\RegionMarkerTok}[1]{#1}
\newcommand{\SpecialCharTok}[1]{\textcolor[rgb]{0.81,0.36,0.00}{\textbf{#1}}}
\newcommand{\SpecialStringTok}[1]{\textcolor[rgb]{0.31,0.60,0.02}{#1}}
\newcommand{\StringTok}[1]{\textcolor[rgb]{0.31,0.60,0.02}{#1}}
\newcommand{\VariableTok}[1]{\textcolor[rgb]{0.00,0.00,0.00}{#1}}
\newcommand{\VerbatimStringTok}[1]{\textcolor[rgb]{0.31,0.60,0.02}{#1}}
\newcommand{\WarningTok}[1]{\textcolor[rgb]{0.56,0.35,0.01}{\textbf{\textit{#1}}}}
\usepackage{graphicx}
\makeatletter
\def\maxwidth{\ifdim\Gin@nat@width>\linewidth\linewidth\else\Gin@nat@width\fi}
\def\maxheight{\ifdim\Gin@nat@height>\textheight\textheight\else\Gin@nat@height\fi}
\makeatother
% Scale images if necessary, so that they will not overflow the page
% margins by default, and it is still possible to overwrite the defaults
% using explicit options in \includegraphics[width, height, ...]{}
\setkeys{Gin}{width=\maxwidth,height=\maxheight,keepaspectratio}
% Set default figure placement to htbp
\makeatletter
\def\fps@figure{htbp}
\makeatother
\setlength{\emergencystretch}{3em} % prevent overfull lines
\providecommand{\tightlist}{%
  \setlength{\itemsep}{0pt}\setlength{\parskip}{0pt}}
\setcounter{secnumdepth}{-\maxdimen} % remove section numbering
\ifLuaTeX
  \usepackage{selnolig}  % disable illegal ligatures
\fi
\IfFileExists{bookmark.sty}{\usepackage{bookmark}}{\usepackage{hyperref}}
\IfFileExists{xurl.sty}{\usepackage{xurl}}{} % add URL line breaks if available
\urlstyle{same}
\hypersetup{
  pdftitle={Exercise2},
  hidelinks,
  pdfcreator={LaTeX via pandoc}}

\title{Exercise2}
\author{}
\date{\vspace{-2.5em}2023-07-22}

\begin{document}
\maketitle

\usepackage{ctex}

\textbackslash begin\{document\}

\hypertarget{r-markdown}{%
\subsection{R Markdown}\label{r-markdown}}

This is an R Markdown document. Markdown is a simple formatting syntax
for authoring HTML, PDF, and MS Word documents. For more details on
using R Markdown see \url{http://rmarkdown.rstudio.com}.

When you click the \textbf{Knit} button a document will be generated
that includes both content as well as the output of any embedded R code
chunks within the document. You can embed an R code chunk like this:

\begin{Shaded}
\begin{Highlighting}[]
\CommentTok{\#Q1}
\FunctionTok{data}\NormalTok{(rivers)}
\CommentTok{\#T1}
\FunctionTok{print}\NormalTok{(rivers)}
\end{Highlighting}
\end{Shaded}

\begin{verbatim}
##   [1]  735  320  325  392  524  450 1459  135  465  600  330  336  280  315  870
##  [16]  906  202  329  290 1000  600  505 1450  840 1243  890  350  407  286  280
##  [31]  525  720  390  250  327  230  265  850  210  630  260  230  360  730  600
##  [46]  306  390  420  291  710  340  217  281  352  259  250  470  680  570  350
##  [61]  300  560  900  625  332 2348 1171 3710 2315 2533  780  280  410  460  260
##  [76]  255  431  350  760  618  338  981 1306  500  696  605  250  411 1054  735
##  [91]  233  435  490  310  460  383  375 1270  545  445 1885  380  300  380  377
## [106]  425  276  210  800  420  350  360  538 1100 1205  314  237  610  360  540
## [121] 1038  424  310  300  444  301  268  620  215  652  900  525  246  360  529
## [136]  500  720  270  430  671 1770
\end{verbatim}

\begin{Shaded}
\begin{Highlighting}[]
\CommentTok{\#T2 method1}
\FunctionTok{length}\NormalTok{(rivers)}
\end{Highlighting}
\end{Shaded}

\begin{verbatim}
## [1] 141
\end{verbatim}

\begin{Shaded}
\begin{Highlighting}[]
\FunctionTok{mean}\NormalTok{(rivers)}
\end{Highlighting}
\end{Shaded}

\begin{verbatim}
## [1] 591.1844
\end{verbatim}

\begin{Shaded}
\begin{Highlighting}[]
\FunctionTok{median}\NormalTok{(rivers)}
\end{Highlighting}
\end{Shaded}

\begin{verbatim}
## [1] 425
\end{verbatim}

\begin{Shaded}
\begin{Highlighting}[]
\FunctionTok{sd}\NormalTok{(rivers)}
\end{Highlighting}
\end{Shaded}

\begin{verbatim}
## [1] 493.8708
\end{verbatim}

\begin{Shaded}
\begin{Highlighting}[]
\FunctionTok{var}\NormalTok{(rivers)}
\end{Highlighting}
\end{Shaded}

\begin{verbatim}
## [1] 243908.4
\end{verbatim}

\begin{Shaded}
\begin{Highlighting}[]
\FunctionTok{min}\NormalTok{(rivers)}
\end{Highlighting}
\end{Shaded}

\begin{verbatim}
## [1] 135
\end{verbatim}

\begin{Shaded}
\begin{Highlighting}[]
\FunctionTok{max}\NormalTok{(rivers)}
\end{Highlighting}
\end{Shaded}

\begin{verbatim}
## [1] 3710
\end{verbatim}

\begin{Shaded}
\begin{Highlighting}[]
\CommentTok{\#T2 method2}
\FunctionTok{library}\NormalTok{(Hmisc)}
\end{Highlighting}
\end{Shaded}

\begin{verbatim}
## 
## 载入程辑包:'Hmisc'
\end{verbatim}

\begin{verbatim}
## The following objects are masked from 'package:base':
## 
##     format.pval, units
\end{verbatim}

\begin{Shaded}
\begin{Highlighting}[]
\FunctionTok{describe}\NormalTok{(rivers)}
\end{Highlighting}
\end{Shaded}

\begin{verbatim}
## rivers 
##        n  missing distinct     Info     Mean      Gmd      .05      .10 
##      141        0      114        1    591.2    428.5      230      255 
##      .25      .50      .75      .90      .95 
##      310      425      680     1054     1450 
## 
## lowest :  135  202  210  215  217, highest: 1885 2315 2348 2533 3710
\end{verbatim}

\begin{Shaded}
\begin{Highlighting}[]
\CommentTok{\#T3}
\NormalTok{rivers.Des}\FloatTok{.1} \OtherTok{\textless{}{-}} \FunctionTok{c}\NormalTok{(}\FunctionTok{length}\NormalTok{(rivers),}\FunctionTok{mean}\NormalTok{(rivers),}\FunctionTok{median}\NormalTok{(rivers),}\FunctionTok{sd}\NormalTok{(rivers),}\FunctionTok{var}\NormalTok{(rivers),}\FunctionTok{min}\NormalTok{(rivers),}\FunctionTok{max}\NormalTok{(rivers));rivers.Des}\FloatTok{.1}
\end{Highlighting}
\end{Shaded}

\begin{verbatim}
## [1]    141.0000    591.1844    425.0000    493.8708 243908.4086    135.0000
## [7]   3710.0000
\end{verbatim}

\begin{Shaded}
\begin{Highlighting}[]
\CommentTok{\#T4}
\NormalTok{river.Des}\FloatTok{.2} \OtherTok{\textless{}{-}} \FunctionTok{as.data.frame}\NormalTok{(}\FunctionTok{describe}\NormalTok{(rivers)}\SpecialCharTok{$}\NormalTok{counts)}
\NormalTok{river.Des}\FloatTok{.2}\SpecialCharTok{$}\NormalTok{FeatureName }\OtherTok{\textless{}{-}} \FunctionTok{row.names}\NormalTok{(river.Des}\FloatTok{.2}\NormalTok{)}
\NormalTok{river.Des}\FloatTok{.2} \OtherTok{\textless{}{-}}\NormalTok{ river.Des}\FloatTok{.2}\NormalTok{[,}\FunctionTok{c}\NormalTok{(}\DecValTok{2}\NormalTok{,}\DecValTok{1}\NormalTok{)] }\CommentTok{\#颠倒了下两个列的顺序}
\FunctionTok{colnames}\NormalTok{(river.Des}\FloatTok{.2}\NormalTok{) }\OtherTok{=} \FunctionTok{c}\NormalTok{(}\StringTok{\textquotesingle{}FeatureName\textquotesingle{}}\NormalTok{, }\StringTok{\textquotesingle{}Values\textquotesingle{}}\NormalTok{)}

\CommentTok{\#Q2}
\FunctionTok{rm}\NormalTok{(}\AttributeTok{list =} \FunctionTok{ls}\NormalTok{()) }\CommentTok{\#清空所有变量}
\FunctionTok{data}\NormalTok{(women)}
\CommentTok{\#T1}
\FunctionTok{print}\NormalTok{(}\FunctionTok{dim}\NormalTok{(women))}
\end{Highlighting}
\end{Shaded}

\begin{verbatim}
## [1] 15  2
\end{verbatim}

\begin{Shaded}
\begin{Highlighting}[]
\CommentTok{\#T2}
\NormalTok{women[}\FunctionTok{c}\NormalTok{(}\DecValTok{1}\SpecialCharTok{:}\DecValTok{6}\NormalTok{),] }\CommentTok{\#前六个}
\end{Highlighting}
\end{Shaded}

\begin{verbatim}
##   height weight
## 1     58    115
## 2     59    117
## 3     60    120
## 4     61    123
## 5     62    126
## 6     63    129
\end{verbatim}

\begin{Shaded}
\begin{Highlighting}[]
\NormalTok{women[}\FunctionTok{c}\NormalTok{((}\FunctionTok{dim}\NormalTok{(women)[}\DecValTok{1}\NormalTok{] }\SpecialCharTok{{-}} \DecValTok{6}\NormalTok{)}\SpecialCharTok{:}\FunctionTok{dim}\NormalTok{(women)[}\DecValTok{1}\NormalTok{]),]}
\end{Highlighting}
\end{Shaded}

\begin{verbatim}
##    height weight
## 9      66    139
## 10     67    142
## 11     68    146
## 12     69    150
## 13     70    154
## 14     71    159
## 15     72    164
\end{verbatim}

\begin{Shaded}
\begin{Highlighting}[]
\CommentTok{\#T3}
\FunctionTok{mean}\NormalTok{(women}\SpecialCharTok{$}\NormalTok{height)}
\end{Highlighting}
\end{Shaded}

\begin{verbatim}
## [1] 65
\end{verbatim}

\begin{Shaded}
\begin{Highlighting}[]
\FunctionTok{var}\NormalTok{(women}\SpecialCharTok{$}\NormalTok{height)}
\end{Highlighting}
\end{Shaded}

\begin{verbatim}
## [1] 20
\end{verbatim}

\begin{Shaded}
\begin{Highlighting}[]
\CommentTok{\#T4}
\NormalTok{women.Height60 }\OtherTok{\textless{}{-}}\NormalTok{ women[}\FunctionTok{which}\NormalTok{(women}\SpecialCharTok{$}\NormalTok{height }\SpecialCharTok{\textgreater{}} \DecValTok{60}\NormalTok{),] }\CommentTok{\#加不加which看起来没区别}
\CommentTok{\#T5}
\NormalTok{women.List }\OtherTok{=} \FunctionTok{as.list}\NormalTok{(women)}
\CommentTok{\#T6}
\NormalTok{women.matrix }\OtherTok{=} \FunctionTok{as.matrix}\NormalTok{(women)}
\FunctionTok{print}\NormalTok{(women.matrix)}
\end{Highlighting}
\end{Shaded}

\begin{verbatim}
##       height weight
##  [1,]     58    115
##  [2,]     59    117
##  [3,]     60    120
##  [4,]     61    123
##  [5,]     62    126
##  [6,]     63    129
##  [7,]     64    132
##  [8,]     65    135
##  [9,]     66    139
## [10,]     67    142
## [11,]     68    146
## [12,]     69    150
## [13,]     70    154
## [14,]     71    159
## [15,]     72    164
\end{verbatim}

\begin{Shaded}
\begin{Highlighting}[]
\FunctionTok{print}\NormalTok{(}\FunctionTok{t}\NormalTok{(women.matrix))}
\end{Highlighting}
\end{Shaded}

\begin{verbatim}
##        [,1] [,2] [,3] [,4] [,5] [,6] [,7] [,8] [,9] [,10] [,11] [,12] [,13]
## height   58   59   60   61   62   63   64   65   66    67    68    69    70
## weight  115  117  120  123  126  129  132  135  139   142   146   150   154
##        [,14] [,15]
## height    71    72
## weight   159   164
\end{verbatim}

\begin{Shaded}
\begin{Highlighting}[]
\CommentTok{\#T7}
\FunctionTok{cor}\NormalTok{(women)}
\end{Highlighting}
\end{Shaded}

\begin{verbatim}
##           height    weight
## height 1.0000000 0.9954948
## weight 0.9954948 1.0000000
\end{verbatim}

\begin{Shaded}
\begin{Highlighting}[]
\CommentTok{\#T8}
\end{Highlighting}
\end{Shaded}

Note that the \texttt{echo\ =\ FALSE} parameter was added to the code
chunk to prevent printing of the R code that generated the plot.

\end{document}
